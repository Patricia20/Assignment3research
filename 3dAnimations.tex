\documentclass[12pt,a4paper]{article}
\author{Giramia Patricia}
\title{3D Animations}
\begin{document}
\maketitle
\textsf{Research methodology BIS 2207
Lecturer: Mr. Ernest Mwebaze
Research topic: Computer animation
Case study: various programs used to create 3d animations
Giramia Patricia
16/u/4831/eve}
\begin{center}
\textbf{
\section*{Introduction}}
\end{center}
\paragraph*{Computer animation }
Computer animation is the process used for generating animated images. The more general term computer-generated imagery (CGI) encompasses both static scenes and dynamic images, while computer animation only refers to the moving images. Modern computer animation usually uses 3D computer graphics, although 2D computer graphics are still used for stylistic, low bandwidth, and faster real-time renderings. Sometimes, the target of the animation is the computer itself, but sometimes film as well.
\begin{center}
\textbf{
\section*{Background}}
\end{center}
\paragraph*{History of computer animations}
Timeline of computer animation in film and television Early digital computer animation was developed at Bell Telephone Laboratories in the 1960s by Edward E. Zajac, Frank W. Sinden, Kenneth C. Knowlton, and A. Michael Noll. From then on, numerous industries adopted to the beauty of 3d animation.
\begin{center}
\textbf{
\section*{Data Collection}}
\end{center}
\begin{center}
\subsection{COMPUTER ANIMATIONS}
This describes the purpose of the form and an introduction to the research topic
\end{center}
\begin{center}
\subsection{The clear"Picture"}
images will help understanding 
\end{center}
\begin{center}
\subsection{Date}
date of filling the form
\end{center}
\begin{center}
\subsection{Name}
name of the participant
\end{center}
\begin{center}
\subsection{Occupation}
what the participant is or does
\end{center}
\begin{center}
\subsection{Address}
place of residence
\end{center}
\begin{center}
\subsection{How is 3D Animation?}
general overview of the topic to the participant
\end{center}
\begin{center}
\subsection{Any support of your opinion?}
reason why the participant would decline supporting the topic, if he/she chose the "Bad" option
\end{center}
\begin{center}
\subsection{Overthrow}
the participant is required to rank 3d animations design according to whatever percentage he/she desires
\end{center}
\begin{center}
\subsection{Preferences}
what the participant would prefer, 3d or 2d animations design
\end{center}
\begin{center}
\subsection{Thank you}
appreciation to the participant for showing interest
\end{center}
\textbf{
\section*{problem Statement}}
\end{center}
\paragraph*{•}
Assessing to what extent 3D Animations have improved the quality of animation design compared to 2D.
\begin{center}
\textbf{
\section*{Objectives}}
\end{center}
\begin{center}
\paragraph*{Major Objectives}
\begin{list}{To assess the features of 3D Animations}{To look at the tools that support 3D Animations }

\end{list}
\paragraph*{Minor Objectives}
\begin{list}{To assess the features of 2D Animations}{To look at the different software that support 2D Animations}

\end{list}
\textbf{
\section*{Scope}}
\end{center}
\paragraph*{About 3D applications}
Sample 3d animations including movie making and modelling respectively.
\begin{figure}[h!]
\includegraphics[width=\linewidth]{pic.PNG}
\caption{1-movie making}
\end{figure}
\\
Some of the major programs or applications that support 3d animations include;
\\
\begin{flushleft}
\subsection{Blender.}
\subparagraph*{About it}
\end{flushleft}
Blender is a freeware and open source 3D image editor which works with modelling, image animation and rendering of 3D graphics formats inside of a clean, customizable interface. With this very well-done editor, users have access to a wide range of powerful editing tools for \textit{creating objects, sculpting and painting objects with various types of textures}. This program is highly capable of creating complex 3D scenes. In the game mode, graphics publishers and game creators are given access to useful tools.
\begin{flushleft}
\subsection{Art of illusion.}
\subparagraph*{About it}
\end{flushleft}
Art of Illusion is a free and open source 3D rendering program that is capable of creating three-dimensional scenes, shapes and animations. The program interface is easy-to-understand, yet intuitive and displays the scene you're working on from different perspectives. All of the app's features are accessed through the various menus and windows. Art of Illusion provides many of the necessary tools to\textit{ create figures, models, textures, surfaces and shapes}. Lighting and shadow effects are easily achieved. Overall, Art of Illusion is an exciting free tool which provides a free way to work with 3D objects with an intuitive interface.
\begin{flushleft}
\subsection{3D Crafter}
\subparagraph*{About it}
\end{flushleft}
3DCrafter is a graphics tool for drawing and animating 3D objects. One of its main features is \textit{drag-and-drop. Formerly known as 3DCanvas}, you can use many of the tools provided in this application to create 3D animations. You can create complex models after learning your way around. The animation here can be frame-by-frame and displayed in real-time. 3DCrafter can also\textit{ create AVI video files} from your work. 3DCrafter is able to import many of the most popular 3D object files and can also import freely available models found on the Internet. The free version is limited in some of its capabilities.
\\
Other software includes:
\\
\begin{list}{	K-3D, Earthquake 3d}{	Easy 3D Objects}{	Easy 3D Objects}{	DirectX}
\item 
\end{list}
\\
Among others.
\\
Sample code used to provide animations to any sprite.
\\
\begin{lstlisting}[language=java]
\emph{var int x := 0, y := 
screenHeight / 2; 
while x < screenWidth 
drawBackground () drawSpriteAtXY (x, y) // draw on top of the background 
x:= x + 5 // move to the right
}
\end{lstlisting}
\begin{center}
\section*{Conclusion}
\end{center}
\\
Computer-generated animations are more controllable than other more physically based processes, constructing miniatures for effects shots or hiring extras for crowd scenes, and because it allows the creation of images that would not be feasible using any other technology.
Developers of computer games and 3D video cards strive to achieve the same visual quality on personal computers in real-time as is possible for CGI films and animation.
All these programs support computer animation in different ways and hence build better characters that contribute a lot to the industries and developers.
\end{document}